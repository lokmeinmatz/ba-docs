\pagecolor{fu-orange}

\section*{Guidelines}
\emph{Please remove this section before handing in your proposal.}

You should plan 3--4 feedback rounds for finalizing your proposal.
Before handing in the proposal to your supervisor, please check if your proposal fulfills the following criteria:

\begin{todolist}
  \item A proposal for a B. Sc. thesis should have about five pages of text. Excluding the title page, the timeline, references, and the outline.
  \item A proposal for an M. Sc. thesis should have a maximum of 10 to 20 pages. Please keep in mind, the text can be used for the thesis document. 
  \item Your proposal has to include a clearly stated research question. The research question has to be in the form of a sentence ending with a question mark.
  \item Include the assumed deadline for submitting your thesis in the timeline.
  \item Define a short, significant title that reflects the contents of your thesis. Please note, you can change the title in the final version.
  \item Use American English, for example, instead of vi\textbf{s}ualisation (BE) $\rightarrow$ visuali\textbf{z}ation (AE).
  \item You can write the proposal and also the final thesis either in German or in English. However, we recommend English. If you use German, please adapt the latex template on your own.
  \item The proposal should be comprehensible, so that not only specialists can understand it.
  \item Use correct spelling and grammar, as well as correct scientific language. Please use ''they'' as a gender neutral pronoun, rather than ''he or she,'' ''he/she,'' or ''s/he''. Using explicitly gendered pronouns excludes non-binary individuals and people who do not use solely binary ''he'' or ''she'' pronouns. For further information, please read Scheuerman et al.~\cite{scheuerman2020:hcigenderguidelines}.
  \item Use the ACM citation style.
  \item Avoid any indication of possible plagiarism. Please remember, in severe cases, plagiarism results in a fail.
 
\end{todolist}

\subsection*{Where to find literature?}
Be aware that the Human-Centered Computing (HCC) Research Group conducts research in the areas of Human-Computer Interaction (HCI) and Collaborative Computing. Therefore, articles from the following conferences are of particular interest\footnote{Upcoming conference are listed on the website of the ACM Special Interest Group on Computer-Human Interaction (CHI): \url{https://sigchi.org/conferences/upcoming-conferences/}}: 

\begin{itemize}
    \itemsep0em % This should move to the global layout section.
    \item CHI --- Human Factors in Computing Systems
    \item CSCW --- Computer Supported Cooperative Work
    \item IUI --- International Conference on Intelligent User Interfaces
    \item DIS --- Designing Interactive Systems Conference
\end{itemize}

You can find the publications of these conferences in the Digital Library of ACM\footnote{ \url{https://dl.acm.org/}}. We highly recommend this search engine, as opposed to using the general search engine Google Scholar\footnote{\url{https://scholar.google.com/}}. In ACM, the papers are already contextualized (by considering the respective conference in the search filters), and thus, you ensure that the work is relevant. If you have any doubts, talk to your supervisor.

\subsection*{What is a human-centered design approach?}
At HCC Research Group, we always apply a human-centered design (HCD) approach. This approach is taught in the \emph{Human-Computer Interaction I} course. If you are not familiar with this approach, you need to learn it before starting your thesis. Please contact your supervisor to get access to the slides and videos of the course. In order to reach your goals, you have to make sure that you understand which research methods are most suitable to fulfill your goals. We recommend reviewing available methods and their practical application in the field of HCI~\cite{lazar2017research, olson2014ways}. Both mentioned sources and additional books are available at the HCC Research Group's library\footnote{We also set up an open Zotero library where we share literature references that have shown to be helpful in prior theses regarding fundamental knowledge on HCI methods. You can find the open Zotero library ''HCC --- Thesis Literature'' here: \url{https://www.zotero.org/groups/2815440/hcc_-_thesis_literature}.}. Please talk to your supervisor.


\afterpage{\nopagecolor}