% ---------------------------------------------------
% ----- Conclusion of the template
% ----- for Bachelor-, Master thesis and class papers
% ---------------------------------------------------
%  Created by C. Müller-Birn on 2012-08-17, CC-BY-SA 3.0.
%  Freie Universität Berlin, Institute of Computer Science, Human Centered Computing. 
%
\chapter{Conclusion and outlook}
\label{chap:conclusion}      
This thesis demonstrates the challenges and opportunities of applying HCI principles and methods in a brownfield software development project through the redevelopment of a UI Editor for a digital publishing company, Sprylab.
During the process, it was shown how common user research methods can be adapted to this specific case study and how technical limits and users needs can be combined in this context.
Agile development and Lean UX enabled iterating quickly with new ideas and achieving fast deployments to a staging system.
A growing group of users at Sprylab already using the software in production and a majority of positive feedback show that the chosen methods and from their outcome derived features were the right approach to enhance the user experience while complying with technical requirements.
Additionally, this thesis can help to argue for the use of HCI methods in future projects started at Sprylab as well.
I'm confident that this new software is a stable and extensible tool, especially regarding the two factors Usability and Time-on-Task.
\\\\
However, there is still room for improvements in terms of accessibility and entry hurdle for new users.
The entry hurdle is still high and a lot of background knowledge is assumed, which is partly due to the complexity of other systems in the ecosystem that were set as technical requirements from the beginning on.
Moving forward, replacing the JSON editor with a custom implementation that combines JSON Schemata and generated UI is one of the next steps, which will speed up the user's
workflow even more and allow for further improvements that are impossible with a third-party library.
\\\\
Overall, this thesis has demonstrated the importance of considering HCI principles and methods in brownfield software development projects, and the potential benefits that can be achieved when these approaches are applied in a real-world context.