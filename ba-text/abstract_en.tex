% !TeX root = thesis_main.tex
% ---------------------------------------------------
% ----- Abstract (English) of the template
% ----- for Bachelor-, Master thesis and class papers
% ---------------------------------------------------
%  Created by C. Müller-Birn on 2012-08-17, CC-BY-SA 3.0.
%  Freie Universität Berlin, Institute of Computer Science, Human Centered Computing. 
%
\pagestyle{empty}

\subsection*{Abstract}

In recent years, the shift from print to digital publishing has increased the need for tools that allow publishers to quickly build and configure apps and websites.
While progress has been made in this area with the development of general purpose website builders and headless content management systems, the internal tools and software used by publishing administrators also need to evolve and improve.
\\\\
The goal of this bachelor thesis is to conceptualize, plan, and implement an UI editor for apps and websites used by publishers, particularly in Germany and the UK.
The editor will be built using a web framework called "Purple Experience," which is used to deliver apps and websites generated from the same configuration and assets to end users.
This service is closely linked to other existing software systems that are used to edit content, manage apps, and deliver content.
\\
This "brownfield" software project presents both challenges and opportunities.
The flexibility of the editor is restricted by existing workflows and software that cannot be changed, but it also has a diverse group of users who have varying levels of experience with these software products.
These users, who include internal framework developers, customer support, and project developers, as well as external people at publishing houses, provide valuable insights into their current workflows and how they believe the editor can improve their productivity and enjoyment.
\\
To gain these insights, the use of existing software was evaluated, and multiple user research methods, including moderated observations and interviews, were applied.
The outcome of this research will be useful as guidance for future software development projects for internal tools at companies,
or in environments where constraints exist but a user base is already in place to provide valuable input and feedback.
\\
The outcome should be usable as guidance for future software development projects for internal tools at companies or in environments where constraints may exist, but also an approachable, already present user base can give valuable input and feedback.
\\\\
Based on the evaulations oth the user research phase, I built an interactive prototype using modern web technologies like react, express.js and Typescript.
This was deployed using continous integration to a controlled group of test users. This allowed to get quick feedback and iterate fast, until the tool can be made available to a broader audience.
\\\\
TODO: the outcomes of the thesis consist of a working software product that is actively used by early adopters, as well 

\cleardoublepage
