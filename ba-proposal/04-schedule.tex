\section{Project Plan}
\label{sec:plan}

It is useful to understand a Bachelor and Master thesis as a project. Projects are based on a plan, and each plan needs milestones\footnote{By milestone we mean a collection of tasks, which need to be finished by a specific date. You can also call it a work package.} and a timeline. Thus, in this section, you will break down your thesis project into manageable and specific milestones to realistically estimate the time you need. Especially if you use methods for the first time, we recommend to discuss this timeline with your supervisor. Please describe each milestone, what do you exactly do in that phase, in what order, what is the result or outcome of each step, and how does it contribute towards the goal of your thesis. As a result, you will outline a detailed timeline for your upcoming research.

According to the exam regulations: a Bachelor thesis\footnote{Please read \S~10 of the Study and Examination regulations for the bachelor’s degree program: \url{https://www.imp.fu-berlin.de/fbv/pruefungsbuero/Studien--und-Pruefungsordnungen/StOPO_BSc_Inf_-2014.pdf}, accessed May 16, 2021} takes about 360~hours (12~LP) and a Master thesis\footnote{Please read \S~9 of the Study and Examination regulations for the master’s
degree program: \url{https://www.imp.fu-berlin.de/fbv/pruefungsbuero/Studien--und-Pruefungsordnungen/STOPO_MSc_-Inf_-2014.pdf}, accessed May 16, 2021} is calculated with 900~hours (30~LP).

\begin{table}[htbp]
\small
\colorbox{usethiscolorhere}{
\centering
\begin{tabularx}{\textwidth}{@{} r Y @{}}
	& \begin{todolist}
  \itemsep0em % This should move to the global layout section.
  \item Calculate the hours you can effectively work on your thesis per week.
  \item Write down the planned date of handing in your thesis.
  \item Include up to 40~\% buffer in case of unforeseen problems (e.g., sickness, vacation).
  \item Include a Gantt-Chart.
\end{todolist}\\
    
\end{tabularx}
}
\end{table}




\subsection{Milestones}
\label{subsec:milestone}
Specify the milestones of your upcoming project. Please describe when you plan to achieve which milestone and what artifact(s) or outcome will result from each milestone. Also, keep in mind what the goal of each milestone is.


\begin{table}[htbp]
\small
\colorbox{usethiscolorhere}{
\centering
\begin{tabularx}{\textwidth}{@{} r Y @{}}
	\textbf{M1}
	& \textbf{Milestone --- Literature and Source Review}\vspace{2mm}\\
	\textbf{Due date} & 2022-09-21 (Week $1$)\vspace{2mm}\\
     \textbf{Tasks} & Identifying and read other studies/thesis/papers about UI editors, HCI methods, bronfield development\\&Look through old HCI lectures\vspace{2mm}\\
    \textbf{Outcome} & A list of relevant papers, articles and possilby open source repositories\\
    & A list of applicable HCI methods and when and with whom they should be applied\\
    & A final text summarizing the main findings and approaches, which might be useful for my project. \vspace{2mm}\\
    \textbf{Goal} & General understanding of methods to build UIs and software with HCI methods in an constrained environment. Having a good foundation for discussing my results in the context of other people's work.\vspace{2mm}\\
    
\end{tabularx}
}
\end{table}

\begin{table}[htbp]
\small
\colorbox{usethiscolorhere}{
\centering
\begin{tabularx}{\textwidth}{@{} r Y @{}}
	\textbf{M2}
	& \textbf{First round of methods \& first deployment of Low-fidelity wireframe prototype}\vspace{2mm}\\
	Due date & 2022-09-30 (Week $3$)\vspace{2mm}\\
     Tasks & Conduct HCI methods to evaulate current state and initial requirements with at least 4 persons from at least 2 diffrent resorts.\\
     & Prepare the codebases for front- and backend including unit tests, some basic UI wireframes to test layouts and diffrent pages\vspace{2mm}\\
    Outcome & A document condensing the outcomes of the survey / observations, including a TODO list derived from their initial requirements and the requirements of the company's software environment.\\
    & A gitlab repository that can build a docker image ready to deploy on kubernetes.\vspace{2mm}\\
    Goal & Having laid the base work for the iterative work on the editor, as well as collected data for the evaluation at the end of the project.\vspace{2mm}\\
    
\end{tabularx}
}
\end{table}


\begin{table}[htbp]
    \small
    \colorbox{usethiscolorhere}{
    \centering
    \begin{tabularx}{\textwidth}{@{} r Y @{}}
        \textbf{M3}
        & \textbf{Deploy editor prototype with full views.json schema support}\vspace{2mm}\\
        Due date & 2022-10-21 (Week $5$)\vspace{2mm}\\
         Tasks & Modify or build an Editor that reads the provided JSON schema for the UI configs and allows modifying the JSON files describing the UI.\\
         & Adapt schema generators to support more metadata / annotations inside the code\vspace{2mm}\\
        Outcome & A editor view where users can manipulate the UI layout for an app.\\
        & Performance should be better than exisitng JSON editor implementations with the usual schema and config sizes (validate using production app clones).\vspace{2mm}\\
        Goal & Having a solid foundation of one of the core features of the editor, to change the UI structure, while beeing more usable than exisitng solutions\vspace{2mm}\\
    \end{tabularx}
    }
\end{table}


\begin{table}[htbp]
    \small
    \colorbox{usethiscolorhere}{
    \centering
    \begin{tabularx}{\textwidth}{@{} r Y @{}}
        \textbf{M4}
        & \textbf{final deployment and final round of evaluation}\vspace{2mm}\\
        Due date & 2022-11-15 (Week $8$)\vspace{2mm}\\
         Tasks & Conduct HCI methods to evaulate current state and initial requirements with at least 4 persons from at least 2 diffrent resorts.\\
         & Prepare the codebases for front- and backend including unit tests, some basic UI wireframes to test layouts and diffrent pages\vspace{2mm}\\
        Outcome & A document condensing the outcomes of the survey / observations, including a TODO list derived from their initial requirements and the requirements of the company's software environment.\\
        & A gitlab repository that can build a docker image ready to deploy on kubernetes.\vspace{2mm}\\
        Goal & Having laid the base work for the iterative work on the editor, as well as collected data for the evaluation at the end of the project.\vspace{2mm}\\
        
    \end{tabularx}
    }
\end{table}


\clearpage
\subsection{Timeline}
\label{subsec:timeline}
    \ganttset{%
        calendar week text={%
            \currentweek
        }%
    }

\begin{ganttchart}[
    hgrid, vgrid, bar label font=\small,
    x unit=1.5mm,
    time slot format=little-endian]{15-9-2022}{8-12-2022}
\gantttitlecalendar{ month=shortname, week=1} \\
    \ganttbar{M1}{15-09-2022}{21-09-2022}\\
    \ganttbar{M2}{22-09-2022}{30-09-2022}\\
    \ganttbar{M3}{1-10-2022}{21-10-2022}\\
    \ganttbar{M4}{22-10-2022}{15-11-2022}\\
    \ganttmilestone{Consultation}{26-09-2022}
    \ganttmilestone{}{31-10-2022}
    \ganttmilestone{}{21-11-2022}\\
    \ganttbar{Writing}{30-9-2022}{5-10-2022}
    \ganttbar{}{22-10-2022}{25-10-2022}
    \ganttbar{}{16-11-2022}{27-11-2022}\\
    \ganttbar{Correcting}{28-11-2022}{1-12-2022}\\
    \ganttbar{Buffer}{2-12-2022}{8-12-2022} \\
    \ganttmilestone{Submission}{8-12-2022}
\end{ganttchart}