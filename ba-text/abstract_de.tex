% !TeX root = thesis_main.tex
% LTeX: language=de-DE
% ---------------------------------------------------
% ----- Abstract (German) of the template
% ----- for Bachelor-, Master thesis and class papers
% ---------------------------------------------------
%  Created by C. Müller-Birn on 2012-08-17, CC-BY-SA 3.0.
%  Freie Universität Berlin, Institute of Computer Science, Human Centered Computing. 
%
% \pagestyle{empty}

\addcontentsline{toc}{chapter}{Zusammenfassung}
\chapter*{Zusammenfassung}

In den letzten Jahren hat die Verlagerung von gedruckten zu digitalen Publikationskanälen dazu geführt, dass Verlage Werkzeuge brauchen, mit denen schnell und einfach digitale Plattformen gebaut und konfiguriert werden können.
\\\\
Die vorliegende Bachelorarbeit befasst sich mit diesem Problem, in dem als Fallstudie ein User Interface (UI) Editor für Apps und Webseiten von Magazin- und Nachrichtenverlagen entworfen und implementiert wird.
Der Editor wird für ein proprietäres Web Framework namens \Gls{experience} gebaut, wobei die Limitierungen des existierenden Softwareökosystems berücksichtigt werden müssen.
Dies brachte Herausforderungen und Einschränkungen für das Design und die Implementierung des Tools mit sich.
\\
Um Einblicke in die Bedürfnisse und Arbeitsweisen der Nutzergruppen zu gewinnen, werden in der Benutzerforschungsphase verschiedene Methoden der Human Computer Interaction (HCI)
Methoden angewandt, darunter moderierte Beobachtungen und Interviews.
\\\\
Die Forschungsergebnisse können als Leitfaden für zukünftige Softwareentwicklungsprojekte
Entwicklungsprojekte für interne Tools in Unternehmen dienen.
Zudem können sie nützlich sein, wenn in der Softwareumgebung zwar Einschränkungen gelten, jedoch eine bereits vorhandene Nutzerbasis wertvollen Input und Feedback liefern kann.
Ergebnis dieser Forschungsphase ist ein interaktiver Prototyp unter Verwendung moderner Web-Technologien. Darauf folgend wird de UI Editor einer Gruppe von Testnutzern zur Verfügung gestellt.
Dieser Ansatz, kombiniert mit agiler Softwareentwicklung, ermöglichte schnelle Iterationen und das Sammeln von direktem Feedback vor dem öffentlichen Release.

Es wird aufgezeigt, wie wichtig die Berücksichtigung von HCI-Prinzipien für Brownfield-Softwareentwicklungsprojekte ist, und ich gebe Beispiele, wie diese Prinzipien in einen realen Kontext integriert werden können.