% !TeX root = thesis_main.tex
% ---------------------------------------------------
% ----- Abstract (English) of the template
% ----- for Bachelor-, Master thesis and class papers
% ---------------------------------------------------
%  Created by C. Müller-Birn on 2012-08-17, CC-BY-SA 3.0.
%  Freie Universität Berlin, Institute of Computer Science, Human Centered Computing. 
%
\pagestyle{empty}

\subsection*{Abstract}

With the shift from print to digital publishing in the magazine and news publisher world in recent years, the needs to quickly build apps and websites and have them configured as easy as possible gained importance.
While companies already progressed in that field with website builders, headless content management systems and more, internal tools and software used by the administrators of the publishing houses also need to evolve and improve over time, as expectations requirements change.
\\\\
The goal if this bachelor thesis is to conceptualize, plan and implement an UI Editor for the mentioned kinds of apps / websites, written inside an company providing a ''digital publishing suite'' to publishers mainly in Germany and the UK.
There, a web framework (called ''Purple Experience'') is used to deliver apps and websites generated from the same configuration and assets to the end users. This service is closely linked to other existing software systems to edit the contents, manage apps and content delivery and more.
\\
This brownfield project provides some special burdens as well as opportunities,
as the flexibility is restricted by exisitng workflows and software, but also a diverse group existing users with different levels of experience with those software products.
They consist of internal framework developers, customer support, project develoeprs or external people at the publishing houses.
These possible future users of the software created for this thesis provided valuable insights
into their current workflows and how they imagine this tool could improve their productivity and be more enjoable to use.
\\
To gain these insights, I evaulated the use and then applied multiple user research methods like moderated observations, interviews and small questionaires.
Due to the limited size of the user group, the goal was not to gain <TODO> with high diversity of their demographics, but to have information saturation from fewer but more valuable insights into peolpe with diffrent workflows.
\\
The outcome should be usable as guidance for future software development projects for internal tools at companies or environments where the product is limited in it's flexibility but should still give the best user experience possible.
\\\\
Based on the evaulations oth the user research phase, I built an interactive prototype using modern web technologies like react, express.js and Typescript.
This was deployed using continous integration to a controlled group of test users. This allowed to get quick feedback and iterate fast, until the tool can be made available to a broader audience.
\\\\
TODO: the outcomes of the thesis consist of a working software product that is actively used by early adopters, as well 

\cleardoublepage
