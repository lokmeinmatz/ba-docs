% !TeX root = thesis_main.tex
% ---------------------------------------------------
% ----- Introduction of the template
% ----- for Bachelor-, Master thesis and class papers
% ---------------------------------------------------
%  Created by C. Müller-Birn on 2012-08-17, CC-BY-SA 3.0.
%  Last upadte: C. Müller-Birn 2015-11-27 
%  Freie Universität Berlin, Institute of Computer Science, Human Centered Computing. 
%
\chapter{Introduction}
\label{chap:introduction}

\section{Topic and context}

In the ever-growing world of software development, many companies are now in the situation to maintain a large software ecosystem with complex dependencies.
Still, there is need for continuous improvement and development to stay competitive.
% but still want to improve their systems by developing new components and tools.
This poses the challenge of improving the software from aspects like user experience, scalability and maintainability while being restricted by the ecosystem.
\\
From my point of view, Greenfield development\footnote{Greenfield- and brownfield development refer to software development concepts, where Greenfield projects start in a new environment and don't have legacy code, while brownfield projects are about upgrading or redeveloping software in an existing environment. \cite{JohnAdamsIt:Greenfield}} is implicitly assumed in the majority of books about HCI.
This approach is not applicable to the situation many software companies are in today.
\\
TODO: rewrite that HCI methods might need different weight / get a bit adapted for brownfield development
However, applying HCI methods for user research and user experience-focused design in brownfield development, where many choices are already made, must be approached differently.

In addition, the three major factors in HCI (Usability, Accessibility and Time-on-task, \Cite[pp. 38-40]{LearnHCI:2020ys}) are often neglected due to tight deadlines and limited resources which usually leads to premature releases and unstable software.
\\\\
My goal was to demonstrate how HCI principles and methods can be applied in a brownfield project, using a real world case study at the company Sprylab as an example.
Sprylab is a company which is engaged in the digital publishing industry, providing SaaS\footnote{Software as a Service, TODO definition} to publishing houses for editing and distributing content.
% Intro UI editor
The case study consists of the redevelopment of a UI Editor which I will describe more in detail in chapter \ref{chap:background} in order to improve user expeirence and usability.

\newpage
\section{Process for research, prototyping and implementation}

While writing the software and thesis, I followed the software design process described in \cite[p. 104]{LearnHCI:2020ys}.
There, the process is divided into three phases: 
\\
\begin{minipage}{\linewidth}
  \begin{itemize}
    \item ''Design Thinking''
    \item ''Lean UX''
    \item ''Agile''
  \end{itemize}
\end{minipage}
\\
For this case study, the abstract plan for the process looks as following:

\begin{figure}
  \includegraphics{pics/process.drawio.png}
  \caption{Software design process for the UI Editor.}
	\label{fig:process}
\end{figure}
