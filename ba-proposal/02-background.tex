% !TeX root = main.tex
%************************************************************************
\section{Background}
\label{sec:background}
% List relevant work by others,
% or preliminary results you have achieved with a detailed and accurate
% explanation and interpretation.
% Include relevant photographs, figures or tables to illustrate the text.
% This section should frame the research questions that your subsequent research will address. 
%************************************************************************
Please add a sentence that summarizes what this section is about and what the reader can expect.
%************************************************************************
\subsection{Context of the Project and Problem Description} 
\label{sec:context}
%************************************************************************
Explain all the surroundings that are necessary to understand the broader context of your work. If necessary, give a brief introduction to non-HCI research literature as background knowledge. It is best to include a specific scenario\footnote{Scenarios are defined as an \emph{''informal narrative description''}. \emph{''It describes human activities or tasks in a story that allows exploration and discussion of contexts, needs, and requirements. It does not necessarily describe the use of software or other technological support used to achieve a goal. Using the vocabulary and phrasing of users means that scenarios can be understood by stakeholders, and they are able to participate fully in development.''}~\cite{preeceInteractionDesignHumancomputer2015}.}. If you are designing a piece of software or graphical user interface, please specify your users and the tasks the users want to perform with your software.

%************************************************************************
\subsection{Related Work}
\label{sec:relatedwork}
%************************************************************************
This section consists of a literature review to situate your thesis in the scientific context. Which academic articles exist in your problem area, and how are they related to your work? When placing your thesis in the context of others, you need to consider other work, which uses a similar methodology or articles, who try to answer similar research questions.

The related work can be split into two (or even three) parts.

\subsubsection{JSON Editor - generative UI}

\begin{itemize}
	\item \href{https://f1000research.com/articles/11-475}{Adamant: a JSON schema-based metadata editor for research data management workflows}
	\item \href{https://json-schema.org/understanding-json-schema/UnderstandingJSONSchema.pdf}{Understanding JSON Schema}
	\item \href{https://www.researchgate.net/publication/220728636_Interactive_model_driven_graphical_user_interface_generation}{Interactive model driven graphical user interface generation }
	\item \href{https://www.sciencedirect.com/science/article/pii/S2352711018300505}{JSON-GUI}
\end{itemize}

Example Implementations which should get evaluated or taken as reference
\begin{itemize}
	\item \url{https://github.com/json-editor/json-editor}
	\item \url{https://jsonforms.io/}
\end{itemize}



\subsubsection{Sources for HCI methods and UI design}

\begin{itemize}
	\item \href{https://www.diva-portal.org/smash/get/diva2:215020/FULLTEXT01.pdf }{Methods and Qualities of a Good User Interface Design}
	\item \href{}{Book: Lern human computer interaction, Christopher Reid Becker}
	\item \href{}{Book: Interaction Design: Beyond Human-Computer Interaction }
\end{itemize}

%************************************************************************
\subsection{Research Questions}
\label{subsec:question}
% Based on your overarching goal and the reviewed research, specify your research question.
%************************************************************************
In this section, you should name your research questions. Your research question should be based on the observation that prior research has a gap and some misconception. You can use words such as \emph{but} or \emph{however} to indicate this. Make sure that your emphasize the significance of your research. 