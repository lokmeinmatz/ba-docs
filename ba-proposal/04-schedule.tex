\section{Project Plan}
\label{sec:plan}

It is useful to understand a Bachelor and Master thesis as a project. Projects are based on a plan, and each plan needs milestones\footnote{By milestone we mean a collection of tasks, which need to be finished by a specific date. You can also call it a work package.} and a timeline. Thus, in this section, you will break down your thesis project into manageable and specific milestones to realistically estimate the time you need. Especially if you use methods for the first time, we recommend to discuss this timeline with your supervisor. Please describe each milestone, what do you exactly do in that phase, in what order, what is the result or outcome of each step, and how does it contribute towards the goal of your thesis. As a result, you will outline a detailed timeline for your upcoming research.

According to the exam regulations: a Bachelor thesis\footnote{Please read \S~10 of the Study and Examination regulations for the bachelor’s degree program: \url{https://www.imp.fu-berlin.de/fbv/pruefungsbuero/Studien--und-Pruefungsordnungen/StOPO_BSc_Inf_-2014.pdf}, accessed May 16, 2021} takes about 360~hours (12~LP) and a Master thesis\footnote{Please read \S~9 of the Study and Examination regulations for the master’s
degree program: \url{https://www.imp.fu-berlin.de/fbv/pruefungsbuero/Studien--und-Pruefungsordnungen/STOPO_MSc_-Inf_-2014.pdf}, accessed May 16, 2021} is calculated with 900~hours (30~LP).

\begin{table}[htbp]
\small
\colorbox{usethiscolorhere}{
\centering
\begin{tabularx}{\textwidth}{@{} r Y @{}}
	& \begin{todolist}
  \itemsep0em % This should move to the global layout section.
  \item Calculate the hours you can effectively work on your thesis per week.
  \item Write down the planned date of handing in your thesis.
  \item Include up to 40~\% buffer in case of unforeseen problems (e.g., sickness, vacation).
  \item Include a Gantt-Chart.
\end{todolist}\\
    
\end{tabularx}
}
\end{table}




\subsection{Milestones}
\label{subsec:milestone}
Specify the milestones of your upcoming project. Please describe when you plan to achieve which milestone and what artifact(s) or outcome will result from each milestone. Also, keep in mind what the goal of each milestone is.


\begin{table}[htbp]
\small
\colorbox{usethiscolorhere}{
\centering
\begin{tabularx}{\textwidth}{@{} r Y @{}}
	\textbf{M1}
	& \textbf{Milestone --- Literature Review}\vspace{2mm}\\
	\textbf{Due date} & 2021-05-26 (Week $2$)\vspace{2mm}\\
     \textbf{Tasks} & Identifying and read other studies/thesis/papers evaluating the usability of chatbots in a medical context\vspace{2mm}\\
    \textbf{Outcome} & A list of relevant papers (e.g. folder in Zotero).\\
    & A written summary for each paper.\\
    & A final text summarizing the main findings and approaches, which might be useful for my project. \vspace{2mm}\\
    \textbf{Goal} & General understanding of methods to evaluate the usability of chatbots in the medical context. Having a good foundation for discussing my results in the context of other people's work.\vspace{2mm}\\
    
\end{tabularx}
}
\end{table}

\begin{table}[htbp]
\small
\colorbox{usethiscolorhere}{
\centering
\begin{tabularx}{\textwidth}{@{} r Y @{}}
	\textbf{M2}
	& \textbf{Milestone --- Evaluate Wikipedia's Advanced Search Interface}\vspace{2mm}\\
	Due date & 2021-06-15 (Week $4$)\vspace{2mm}\\
     Tasks & Prepare, conduct, and evaluate a remote usability test with four participants.\vspace{2mm}\\
    Outcome & Moderator script for conducting the usability test.\\
    & Affinity diagram with thematic clusters and headlines.\\
    & A list of usability issues sorted by severity.\vspace{2mm}\\
    Goal & Understanding the drawbacks of the current Wikipedia advanced search in order to (re)design a new interface.\vspace{2mm}\\
    
\end{tabularx}
}
\end{table}

\begin{table}[htbp]
\small
\colorbox{usethiscolorhere}{
\centering
\begin{tabularx}{\textwidth}{@{} r Y @{}}
	\textbf{M...}
	& \textbf{Milestone ---  ...}\vspace{2mm}\\
    Due date & \mbox{} \vspace{2mm}\\
    Tasks & \mbox{} \vspace{2mm}\\
    Outcome & \mbox{} \vspace{2mm}\\
    Goal & \mbox{} \vspace{2mm}\\
\end{tabularx}
}
\end{table}

\begin{table}[htbp]
\small
\colorbox{usethiscolorhere}{
\centering
\begin{tabularx}{\textwidth}{@{} r Y @{}}
	\textbf{M5}
	& \textbf{Milestone ---  High-Fidelity Prototype}\vspace{2mm}\\
	Due date & 2021-08-15 (Week $15$)\vspace{2mm}\\
     Tasks & Implement the final design and the main features with HTML and CSS. \vspace{2mm}\\
    Outcome & Repository with code and data on GitLab.\\
    & Deployed on Heroy and public link.\vspace{2mm}\\
    Goal & Interactive prototype, which is deployed and ready for testing.\vspace{2mm}\\
\end{tabularx}
}
\end{table}

\clearpage
\subsection{Timeline}
\label{subsec:timeline}
Now you need to transfer the milestones into a timeline. The time for your thesis will help you to set realistic time goals and maybe reconsider milestones. Use a Gantt chart\footnote{Check out Wikipedia for an extended overview of project management software that fits your needs: \url{https://en.wikipedia.org/wiki/Comparison_of_project_management_software}. For example Ganttproject is free of charge and open source: \url{https://www.ganttproject.biz/}, accessed: May 26, 2021} for visualization. Please consider what tasks you can do in parallel. Also indicate how you will handle the writing process within your timeline.

    \ganttset{%
        calendar week text={%
            \currentweek
        }%
    }

\begin{ganttchart}[
    hgrid, vgrid, bar label font=\small,
    x unit=1.5mm,
    time slot format=little-endian]{24-5-2021}{15-8-2021}
\gantttitlecalendar{ month=shortname, week=1} \\
    \ganttbar{M1}{24-05-2021}{6-06-2021}\\
    \ganttbar{M2}{7-06-2021}{20-06-2021}\\
    \ganttmilestone{Presentation}{20-06-2021} \\
    \ganttbar{M3}{15-06-2021}{5-7-2021}
    \ganttmilestone{}{5-07-2021} \\
    \ganttbar{Writing}{6-6-2021}{8-6-2021}
    \ganttbar{}{12-7-2021}{27-7-2021}\\
    \ganttbar{Correcting}{1-08-2021}{6-08-2021}
    \ganttmilestone{}{6-08-2021} \\
    \ganttbar{Buffer}{7-08-2021}{14-08-2021} \\
    \ganttmilestone{Submission}{15-08-2021}
\end{ganttchart}