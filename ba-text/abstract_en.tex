% !TeX root = thesis_main.tex
% ---------------------------------------------------
% ----- Abstract (English) of the template
% ----- for Bachelor-, Master thesis and class papers
% ---------------------------------------------------
%  Created by C. Müller-Birn on 2012-08-17, CC-BY-SA 3.0.
%  Freie Universität Berlin, Institute of Computer Science, Human Centered Computing. 
%

\chapter{Abstract}

In recent years, the shift from print to digital publishing channels has increased the need for tools that allow publishers to quickly build and configure apps and websites.
\\\\
The goal of this bachelor thesis is to conceptualize, plan and implement an User Interface (UI) editor for apps and websites used by magazine and news publishers, particularly in Germany and the UK.
The editor was built for a proprietary web framework called "Purple Experience" and was developed within the constraints of an existing software ecosystem, which posed challenges and limitations on the design and implementation of the tool.
\\
To gain insights into the needs and workflows of the groups of users who will be using the editor, a variety of Human Computer Interaction (HCI) methods were applied during the user research phase, including moderated observations and interviews. 
\\\\
The outcome of this research is useful as guidance for future software development projects for internal tools at companies. It can also be useful in environments where constraints exist, but a user base is already in place to provide valuable input and feedback.
As a result of the user research phase, an interactive prototype was built using modern web technologies.
In the next step it was deployed to a controlled group of test users.
This approach combined with methods of agile software development allowed me to iterate fast and collect direct feedback until it was rolled out for production use.

\textbf{TODO reformulate}
\\
Learned: 
- if existing user base (manageable number): qualitative research methods seem to provide more value / deeper feedback than quantitative
\\
- if existing products: use tracking etc. to figure out who already uses tools how, as a base line for comparison
\\
- use vizualizations and other methods to sort ideas by multiple dimensions (user value, impl. effort) -> easier to reason what to do next \& to convince other stakeholders