% !TeX root = thesis_main.tex
% ---------------------------------------------------
% ----- Introduction of the template
% ----- for Bachelor-, Master thesis and class papers
% ---------------------------------------------------
%  Created by C. Müller-Birn on 2012-08-17, CC-BY-SA 3.0.
%  Last upadte: C. Müller-Birn 2015-11-27 
%  Freie Universität Berlin, Institute of Computer Science, Human Centered Computing. 
%
\chapter{Introduction}
\label{chap:introduction}

Im folgenden werden Ihnen Hinweise zur Strukturierung und zum Inhalt des ersten Kapitels gegeben.

\section{Topic and context}

In the evergrowing world of software companies, many once startups or companies who started with greenfield development now are in the situation where they maintain a large software ecosystem and have potentially many dependent users,
but still want to improve their systems by developing new components and tools.

This poses the challenge of improving the software from \textbf{TODO: definition of good software?}, while beeing restricted by the ecosystem.
Thus, applying HCI methods for user research and user experience focused design might be limited or need to be approached in a diffrent way then greenfield development.
Also, the common problem of tight deadlines and limited resources tend to lead to premature and unstable software.
Instead of developing software to maximize the three HCI factors \textbf{See HCD principles / factors src, anme the tree?} for the actual users, often ideas from individual stakeholders like the executive floor are realized without adding real value.

On the other hand, having an exsisting user base which works with exisitng tools is a great fundament to evaluate what ''real users'' need. So HCI methods applied to them can yield more helpful and focused results. \textbf{rephrase}

Many of the resources or literature about HCI seem to assume a mostly free degree while developing new tools, and also assume a wide user base with diverse demographic features \textbf{is this valid engilsh?}.
\\
\textbf{List literature srces that dont speak about brownfield development?}
\textbf{Find a contra example that does cope with brownfield}
	

\section{Zielsetzung der Arbeit}
\begin{itemize}
	\item Was sind die mit dieser Arbeit verfolgten Ziele? Welches Problem soll gelöst werden?
	\item Eine Beschreibung der ersten Ideen, der vorgeschlagene Ansatz und die aktuell erreichten Resultate 
	\item Eine Beschreibung, welchen Beitrag die Arbeit leistet, um das vorgestellte Problem zu lösen
	\item Eine Diskussion, wie die vorgeschlagene Lösung sich von bestehenden unterscheidet, was ist neu oder besser?
\end{itemize}

\section{Vorgehen bei der Umsetzung}
\begin{itemize}
	\item Wie will ich meine Ziele erreichen? (Methodische Überlegungen)
	\item Darstellung zum Forschungsdesign.
	\item Insbesondere bei Master: Wie kann die Zielerreichung ``gemessen'' werden? 
\end{itemize}  	

\section{Aufbau der Arbeit}
\begin{itemize}
	\item Welche Schritte werden durchlaufen, um die Ziele zu erreichen?
	\item An dieser Stelle ist beispielsweise eine Grafik hilfreich, um den Aufbau der Arbeit und welche Ergebnisse/Erkenntnisse wo genutzt werden, zu visualisieren. 
	\item Ebenfalls sollten noch Anmerkungen zur Gestaltung der Arbeit gegeben werden, vor allem, da in vielen deutschen Arbeiten englische Fachbegriffe verwendet werden. Ein solcher Text könnte folgendermaßen lauten: 
		\begin{itemize}
			\item ``Abschließend sind hier noch eine Anmerkungen zur Gestaltung der vorliegenden Arbeit. Für die im Folgenden verwendeten personenbezogene Ausdrücke wurde, um die Lesbarkeit der Arbeit zu erhöhen, die männliche Schreibweise gewählt. Des Weiteren werden eine Reihe von englischen Bezeichnungen verwendet, um einerseits dem interessierten Leser das Studium der häufig vorliegenden englischen Originalliteratur zu erleichtern oder andererseits bestehende Fachbegriffe nicht durch die Übersetzung zu verfälschen. Diese Begriffe sind vom herkömmlichen Text in kursiver Schrift unterschieden.''
		\end{itemize}
\end{itemize}

\begin{figure}[!ht]
	% Mit [!h] wird die Position der Grafik bestimmt. So bedeutet h=here und mit dem "!" (Ausrufezeichen) wird dieser Befehl verstärkt. Weitere Möglichkeiten sind : t=top und b=bottom. Zumeist wird angegeben, in welcher Reihenfolge LaTeX versuchen soll das Bild einzufügen, z.B. [!htb].
	\centering
		\includegraphics[width=0.95\textwidth]{pics/structure.pdf}
	\caption[Beispiel einer möglichen Darstellung zum Aufbau der Arbeit]{Beispiel einer möglichen Darstellung zum Aufbau der Arbeit (vgl. Beschreibung Abschnitt  \ref{chap:chapters}).} 
	% Mit Hilfe von caption wird die Bildunterschrift erzeugt. Der Text in geschweiften Klammern erscheint im Text, während der Text in eckigen Klammern sich dann empfiehlt, wenn die Beschreibung besonders lang ist, denn diese wird dann im Bildverzeichnis verwendet. Diese Kurzbeschreibung kann auch weggelassen werden. 
	\label{fig:structurethesis}
\end{figure}